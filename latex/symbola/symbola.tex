\documentclass[paper=a6,11pt]{scrbook}
\usepackage{geometry}

\usepackage{iftex}

%---------------------------- font settings
\usepackage[utf8]{inputenc}
\usepackage[english,latin.classic]{babel}


\usepackage[T1]{fontenc}

\usepackage{lmodern}
\usepackage{ebgaramond}
\usepackage{lettrine}

\usepackage[protrusion=false,final,letterspace=-100]{microtype}
\newcommand{\spacedtext}[1]{\ifXeTeX{\addfontfeature{LetterSpace=20}\scshape #1} \else \textls[170]{\scshape #1}\fi}

\newcommand{\longs}{\ifXeTeX ſ\else s\fi}

\usepackage{float}
\usepackage{ragged2e}
\usepackage{trimspaces}

\input Zallman.fd


%--------------------------- other packages
\usepackage{hyperref}
\usepackage{marginnote}
\usepackage{graphicx}
\usepackage{parskip}
\usepackage{xcolor}
\usepackage{tikz}
%--------------------------------

\title{\emph{Symbola Aureae Mensae} (1617)}
\author{\LaTeX{} Ninja}
\date{October 2018}

\begin{document}

\maketitle

\section*{Introduction} 
Please go to next page for an example page inspired by the  \href{https://github.com/raphink/geneve_1564}{\emph{Bible de Genève} (1564)} Garamond typesetting.

Looks nicer if compiled using XeLaTeX.

\newpage

\phantom{ }

\begin{figure}[H]
\begin{tikzpicture}[fill opacity=0.5]
\hspace*{-0.5cm}\includegraphics[width=1.15\textwidth]{decoration-vector.png}
\end{tikzpicture}
\end{figure}


\Centering
{\bfseries \textsc{{\Huge SYMBOLA AVREÆ} \linebreak {\Large MENSÆ DVODECIM NA-}\linebreak {\large TIONVM.}}} 

{\large\textit{LIBER PRIMVS.}}

\medskip


\justify
\fontsize{9.5pt}{11pt}\selectfont

\renewcommand{\LettrineFontHook}{\usefont{U}{Zallman}{xl}{n}\color{red!50!black}}
\lettrine[lines=5,,loversize=-0.05, lraise=.05]{P}{}
\emph{O\longs t horribiles minas \& graui\longs \longs imas debac-}
   \marginnote{\fontsize{6.5pt}{7pt}\selectfont\emph{Quæ cau\longs a} \linebreak \emph{occa\longs io} \linebreak \emph{huius con-} \linebreak \emph{uentus Phi-} \linebreak \emph{lo\longs ophici.} } \linebreak
\emph{chationes, quas Pyrgopolynices in} \spacedtext{Che-} \linebreak \spacedtext{miam}  \emph{Virginem, orphanam, \& ab om- \linebreak nibus derelictam, iam pridem effuderat} \linebreak
(proclamans illam \longs puriam, meretri- \linebreak cem, adulteram, deformem, imbellem, impoten- \linebreak tem, mendacem, mendicam, inhone\longs tam, imò o- \linebreak mnium \longs cele\longs ti\longs \longs imam, \longs acrilegam, relegandam \linebreak aut morte pe\longs \longs ima delendam) \emph{hæc præ dolore animi} \linebreak
\emph{impatiens Comitia indixit omnibus, quotquot \longs e\longs e }{\spacedtext{Do-} \linebreak \spacedtext{minam} }\emph{ agno\longs cerent, præ\longs ertim } {\spacedtext{Senioribvs}}\emph{, qui ef-
\linebreak frenatæ illius in\longs aniæ re\longs ponderent, \longs e\longs e\ifXeTeX q́\else q\fi; pro virili ca\longs tam
\linebreak \& intactam defenderent.}
Conuentuitaque habito ex \linebreak præcipuis no\longs tri Orbis nationibus, duodecim in- \linebreak primis comparuerunt, \longs ingulæ in \longs uas {\spacedtext{Tribvs}} 
\end{document}

